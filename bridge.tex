% !TEX root = deckblatt4.tex
\section{Br\"uckengleichrichter}
\subsection{Aufgabenstellung}
In dieser Aufgabe musste ein Br\"uckengleichrichter aufgebaut werden und Zeit- und Frequenzmessungen durchgef\"uhrt werden. Des weiteren sollte eine Fourierreihe Berechnet und mit den Ergebnissen verglichen werden.

\subsection{Fouriereihe}
\begin{center}
  \begin{align*}
    \text{Summensatz S8: } & 2sin(\alpha)cos(\beta) = sin(\alpha-\beta)+sin(\alpha+\beta) \\ \\
    a_0 &= \frac{1}{\pi}\int_0^{2\pi} |sin(t)| dt = \frac{2}{\pi}\int_0^{\pi} sin(t) dt = \frac{4}{\pi} \\ \\
    a_n &= \frac{2}{\pi}\int_0^{\pi} sin(t) * cos (nt) dt \\
    a_n &= \frac{1}{\pi}\int_0^{\pi} sin(t - nt) dt +  \frac{1}{\pi}\int_0^{\pi} sin(t + nt) dt\\
    a_n &= -\frac{1}{\pi} \left[ \frac{cos(\pi(1-n))}{1-n} + \frac{cos(\pi(1+n))}{1+n} \right] \\
    a_n &= -\frac{1}{\pi} \left[ \frac{-(-1)^n-1}{1-n} + \frac{-(-1)^n-1}{1+n} \right] \\
    a_n &= \frac{1}{\pi} \left[ \frac{(-1)^n+1}{1-n} + \frac{-(-1)^n+1}{1+n} \right] \\
    a_n &= \frac{2((-1)^2+1)}{\pi(1-n^2)} \\ \\
    a_n &= \left\{\begin{array}{ll}
            0                     \hspace{2.5cm} \text{f\"ur n ungerade}\\
            \frac{4}{\pi(1-n^2)} \hspace{1.5cm} \text{f\"ur n gerade}
            \end{array}\right.
  \end{align*}
\end{center}
