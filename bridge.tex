% !TEX root = deckblatt4.tex
\section{Br\"uckengleichrichter}
\subsection{Aufgabenstellung}
In dieser Aufgabe musste ein Br\"uckengleichrichter aufgebaut werden und Zeit- und Frequenzmessungen durchgef\"uhrt werden. Des weiteren sollte eine Fourierreihe Berechnet und mit den Ergebnissen verglichen werden.

\subsection{Messschaltung}
\begin{figure}[ht!]
  \begin{center}
    \begin{circuitikz}\draw
    (0,0) to[sI] (0,6) to[Do] (7,6) to[R={$R_1$}{$=1M$}](7,0) to[Do] (0,0)
    (2,0) to[short, *-] (2,4) -- (2.5,4) to[Do] (4.5,4) -- (5,4) to[short,-*] (5,6)
    (5,0) to[short, *-] (5,2) -- (4.5,2) to[Do] (2.5,2) -- (1.5,2) to[short, -*] (1.5,6)
    (0,0) node[ground]{};
    \draw[-latex] (8.5,6) -- (7.3,6);
    \draw (9.6,6) node[] {Kanal 1};
    \draw[-latex] (8.5,0) -- (7.3,0);
    \draw (9.6,0) node[] {Kanal 2};
    \draw (3.5,7) node[] {1N4148};
    \end{circuitikz}
  \end{center}
  \caption{Messschaltung}\label{bsp4_circ}
\end{figure}
\noindent
In Abbildung \ref{bsp4_circ} ist der Br\"uckengleichrichter, welcher f\"ur die nachfolgenden Messungen verwendet wird zu sehen. Der Gelichrichter besteht aus vier Dioden (1N4148). Die Ausgangsspannung am Widerstand $R_1$ kann nicht einfach abgegriffen werden da sonst ein Teil der Schaltung \"uber die Masse des Oszilloskops kurgeschlosswern werden w\"urde. Aus diesem Grund wurde mit 2 Kan\"alen gemessen und die Masse der Tastk\"opfe wurde mit der Masse des Funktionsgenerators an einen Punkt geschalten. Anschlie\ss{}end die Differenz mit der "Math"-Funktion des Oszilloskops berrechnet.

\subsection{Messung im Zeitbereich}
\begin{figure}[H]
 \begin{center}
  \includegraphics[height=6cm,width=12cm]{OsziBilder/bsp4_sin_time_2Vpp_UeUa.png}
 \end{center}
 \caption{Sinussingal $2V_{pp}$, $1kHz$}\label{bsp4_time2V}
\end{figure}
\noindent

\begin{figure}[H]
 \begin{center}
  \includegraphics[height=6cm,width=12cm]{OsziBilder/bsp4_time_10Vpp_UeUa.png}
 \end{center}
 \caption{Sinussingal $10V_{pp}$, $1kHz$}\label{bsp4_time10V}
\end{figure}
\noindent
In den beiden Abbildungen \ref{bsp4_time2V} und \ref{bsp4_time10V} sind die Zeitsignale zu sehen. Die Differenz der beiden Kan\"ale ergibt den Gleichgereichteten Sinus, dieser hat die Doppelte Frequnez als das Eingangssignal, jedoch den gleichen Effektiefwert.

\subsection{Fouriereihe}
\begin{center}
  \begin{align*}
    |sin(t)| & \text{ ist eine gerade Funktion} \\ \\
    S(f) &= \frac{a_0}{2} + \sum_{n=1}^{\infty} a_n * cos(nt) \\ \\
    \text{Summensatz S8: } & 2sin(\alpha)cos(\beta) = sin(\alpha-\beta)+sin(\alpha+\beta) \\ \\
    a_0 &= \frac{1}{\pi}\int_0^{2\pi} |sin(t)| dt = \frac{2}{\pi}\int_0^{\pi} sin(t) dt = \frac{4}{\pi} \\ \\
    a_n &= \frac{2}{\pi}\int_0^{\pi} sin(t) * cos (nt) dt \\
    a_n &= \frac{1}{\pi}\int_0^{\pi} sin(t - nt) dt +  \frac{1}{\pi}\int_0^{\pi} sin(t + nt) dt\\
    a_n &= \frac{1}{\pi} \left[ \frac{cos(\pi(1-n))}{n-1} - \frac{cos(t(1+n))}{n+1} - \frac{1}{n-1} + \frac{1}{n+1} \right] \\
    a_n &= \frac{1}{\pi} \left[ -\frac{cos(\pi(1-n))}{n-1} + \frac{cos(t(1+n))}{n+1} - \frac{1}{n-1} + \frac{1}{n+1} \right] \\
    a_n &= \frac{1}{\pi} \left[ \frac{-(-1)^n (n+1) + (-1)^n (n-1) - (n+1) + (n-1)}{n^2-1} \right] \\
    a_n &= -\frac{1}{\pi} \left[ \frac{2(-1)^n + 1}{\pi(n^2-1)} \right] \\
    a_n &= \left\{\begin{array}{ll}
            0                     \hspace{2.5cm} \text{f\"ur n ungerade}\\
            \frac{4}{\pi(n^2-1)} \hspace{1.5cm} \text{f\"ur n gerade}
            \end{array}\right. \\
    |sin(\omega t)| &= \frac{2}{\pi} + \sum_{n=1}^{\infty}\frac{4}{\pi(4n^2-1)} cos(2n\omega t)
  \end{align*}
\end{center}

\begin{figure}[H]
  \begin{center}
    \begin{tabular}{|c|c|} \hline
    $n$ & $V_{RMS}$ \\ \hline
    $1$ & $1,5005V$ \\ \hline
    $2$ & $300,1mV$ \\ \hline
    $3$ & $128,6mV$ \\ \hline
    $4$ & $71,5mV$ \\ \hline
    $5$ & $45,5mV$ \\ \hline
    \end{tabular}
  \end{center}
  \caption{Berechneten Amplituden der f\"unf ersten Spektralanteile} \label{bsp4_SpecCalc}
\end{figure}



\subsection{FFT}
\begin{figure}[H]
 \begin{center}
  \includegraphics[height=6cm,width=12cm]{OsziBilder/bsp4_sin_fft_10Vpp_dB.png}
 \end{center}
 \caption{Sinussingal $2V_{pp}$, $10kHz$}\label{bsp4_fft}
\end{figure}
\noindent
Genau wie in den Berechnungen ist in dieser Messung zu erkennen, dass jeder zweite Frquenzanteil mit quasi 0 ergibt (ca. $-30dB$) und die Amplituden mit steigender Frequnez rasch kleiner werden.

\begin{figure}[H]
  \begin{center}
    \begin{tabular}{|c|c|c|} \hline
    $n$ & $V_{RMS}$ & Fehler[\%] \\ \hline
    $1$ & $1,325V$    & $11,7\%$\\ \hline
    $2$ & $231,25mV$  & $22,9\%$\\ \hline
    $3$ & $75mV$      & $41,7\%$\\ \hline
    $4$ & $31,25mV$   & $56,3\%$\\ \hline
    $5$ & $12,2mV$    & $73,2\%$\\ \hline
    \end{tabular}
  \end{center}
  \caption{Messwerte der f\"unf gr\"o\ss{}ten Spektralen Komponenten} \label{bsp4_SpecMess}
\end{figure}
\noindent
In Tabelle \ref{bsp4_SpecMess} wurden die ersten f\"unf gro\ss{}en Spektralanteile mithilfe der Cursor Funktion gemessen und mit den in Tabelle \ref{bsp4_SpecCalc} berechneten Werten verglichen. Wie unschwer zu erkennen ist, ist der Fehler bereits beim ersten Spektralanteil mit $11,7\%$ sehr gro\ss{}, dies ist unter anderem mit dem Spannunsabfall an den Dioden in der Gleichrichter Schaltung zu begr\"unden. Je h\"oher der die Freqnuenz desto mehr unterscheiden sich die gemessenen von den berechneten Werten. Der f\"unfte Anteil hat mit einem Fehler von $73,2\%$ bereits nurmehr sehr wenig mit der berechneten Fourierreihe zu tun. 
